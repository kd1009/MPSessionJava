\chapter{Design}
\label{ch:design}

\section{Design aims}

The design of Multiparty SessionJava (MPSJ) required taking into consideration different aspects of software engineering. The objective of providing a framework that can be easily utilised by the end-user often contrasted with the current implementation of SJ. The introduction of multiparty session types, referred to as \textit{global types}, in conjunction with new methods, invoked design considerations in terms of restricting the amount of methods called within the \LST{.sj} programs vs. increasing the amount of calls in the background.

In summary, the following list provides the main design aims developed in the initial stage of the project:

\begin{itemize}
\item Safe and type-checked structured interaction tools for multiparty sessions
\item Simple syntax for global types and general ease of programming
\item Overall efficiency and fast speed of interactions
\item Efficient integration into existing SJ code base, whilst mainting current functionality
\end{itemize}

\section{Syntax}

\subsection{Global Protocol Declarations}

\subsection{Programming in Multi-Party SJ}



\section{Compiler Design Modifications}

\subsection{Extending the Parser Generator}

\subsection{Modifying the AST Node Structure}

\subsection{Global Protocol as the Core of Interaction}



\section{Runtime Design Alterations}

\subsection{Use of Session Participants as SJSockets}

\subsection{Creating the Multi-Party Session}