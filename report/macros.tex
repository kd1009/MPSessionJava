%%%%%%%%%%%%%%%%%%%%%%%%%%%%%%%%%%%%%%%%%%%%%%%%%%%%%%%%%%%%%%%%%%%%%%%%%%%%%
% Latex.
%%%

\newcommand{\REF}[1]{\S\,\ref{#1}} % Section reference. Doesn't add a breaking space in the middle.
\newcommand{\HLINE}{\rule{\linewidth}{.5pt} \\} 


%%%%%%%%%%%%%%%%%%%%%%%%%%%%%%%%%%%%%%%%%%%%%%%%%%%%%%%%%%%%%%%%%%%%%%%%%%%%%
% General text.
%%%

\newcommand{\MYPARAGRAPH}[1]{\vspace{1mm} \noindent \textbf{#1}} % No ending indent.

\newcommand{\ESJ}{eSJ}

\newcommand{\COMMENT}[1]{\noindent [#1]} % Author comment.
\newcommand{\COMMENTNL}[1]{\COMMENT{#1} \\} % Author comment with new line..
\newcommand{\NEWL}{~\\} % Force newline.
\newcommand{\NOTE}{\textbf{N.B.}} % Note.

\newcommand{\RH}[1]{\COMMENT{\textbf{RH}: #1}} % Author comment.
\newcommand{\RHNL}[1]{\COMMENTNL{\textbf{RH}: #1}} 


%%%%%%%%%%%%%%%%%%%%%%%%%%%%%%%%%%%%%%%%%%%%%%%%%%%%%%%%%%%%%%%%%%%%%%%%%%%%%
% Acknowledgements.
%%%

\makeatletter
\newcommand\ackname{Acknowledgements}
\if@titlepage
  \newenvironment{acknowledgements}{%
      \titlepage
      \null\vfil
      \@beginparpenalty\@lowpenalty
      \begin{center}%
        \bfseries \ackname
        \@endparpenalty\@M
      \end{center}}%
     {\par\vfil\null\endtitlepage}
\else
  \newenvironment{acknowledgements}{%
      \if@twocolumn
        \section*{\abstractname}%
      \else
        \small
        \begin{center}%
          {\bfseries \ackname\vspace{-.5em}\vspace{\z@}}%
        \end{center}%
        \quotation
      \fi}
      {\if@twocolumn\else\endquotation\fi}
\fi
\makeatother

%%%%%%%%%%%%%%%%%%%%%%%%%%%%%%%%%%%%%%%%%%%%%%%%%%%%%%%%%%%%%%%%%%%%%%%%%%%%%
% Code.
%%%%

\newcommand{\ARG}[1]{{\rm\em #1}}

%\newcommand{\CODESIZE}{\footnotesize} % Doesn't work: llncs.cls redefines footnotesize as small.
\newcommand{\CODESIZE}{\small} % For "in line" code.
\newcommand{\LISTINGSIZE}{\fontsize{8}{10}\selectfont} % For listings. Apparently, the second argument should be 1.2 times the first.
\newcommand{\CODE}[1]{\texttt{\CODESIZE #1}} % Inline code listing (no extra formatting).
\newcommand{\LST}[1]{\lstinline@#1@} % Inline code listing (with formatting).

\newcommand{\OPIN}[2]{\ensuremath{\CODE{#1}(#2)}} % For application of (input-based) macros.
\newcommand{\OPOUT}[2]{\ensuremath{\CODE{#1}\langle #2 \rangle}} % For application of (output-based) macros.
\newcommand{\PARTY}[1]{\ensuremath{\textbf{#1}}} % Agent names.

\newcommand{\LISTINGSTYLE}{\ttfamily\LISTINGSIZE}

\lstset
{
	language=Java,
	float=hbp,
	%basicstyle=\ttfamily\small,
	basicstyle=\ttfamily\CODESIZE, % For "in line" code as a default. Floated listing figures will need to override this property.
	identifierstyle=\color{Black},
	keywordstyle=\bfseries\color{blue},
	stringstyle=\color{Violet},
	commentstyle=\itshape\color{RedViolet},
	columns=flexible,
	tabsize=4,
	frame=none,
	extendedchars=true,
	showspaces=false,
	showstringspaces=false,
	numbers=none,
	numberstyle=\tiny,
	breaklines=true,
	breakautoindent=true,
	xleftmargin=10pt,
	captionpos=b,
	mathescape=true,
	morekeywords=
	{	
		noalias,
		protocol, begin, cbegin, sbegin, rec,
		accept,
		request, send, pass, copy, receive, recurse, 
		outbranch, inbranch, outwhile, inwhile, recursion, 
		arrived, registerAccept, registerInput, register, select, 
		typecase, when, 
		outsync, insync, outlabel, inlabel, 
		SJTypeable, SJProtocol, SJChannel, SJSocket, SJService, SJServerSocket, SJPort, 
		SJIOException, SJIncompatibleSessionException, 
		SJSelector,  
		sessrec, 
		using,
                selT,
		SJTransport,  SJConnection,  SJConnectionAcceptor,  SJLocalConnection,
		global_protocol,
		setLocal, setRemote, invite, acceptInvite,
		sendInt, receiveInt,
		awaitBegin
	},
	escapeinside={(*@}{@*)}
}

%	identifierstyle=\color{Black},
%	keywordstyle=\bfseries\color{blue},
%	stringstyle=\color{Violet},
%	commentstyle=\itshape\color{RedViolet},


%%%%%%%%%%%%%%%%%%%%%%%%%%%%%%%%%%%%%%%%%%%%%%%%%%%%%%%%%%%%%%%%%%%%%%%%%%%%%
% Hyperrefs.
%%%%

\hypersetup{
    bookmarks=true,         % show bookmarks bar?
    unicode=false,          % non-Latin characters in Acrobat’s bookmarks
    pdftoolbar=true,        % show Acrobat’s toolbar?
    pdfmenubar=true,        % show Acrobat’s menu?
    pdffitwindow=false,     % window fit to page when opened
    pdfstartview={FitH},    % fits the width of the page to the window
    pdftitle={My title},    % title
    pdfauthor={Author},     % author
    pdfsubject={Subject},   % subject of the document
    pdfcreator={Creator},   % creator of the document
    pdfproducer={Producer}, % producer of the document
    pdfkeywords={keywords}, % list of keywords
    pdfnewwindow=true,      % links in new window
    colorlinks=true,       % false: boxed links; true: colored links
    linkcolor=blue,          % color of internal links
    citecolor=PineGreen,        % color of links to bibliography
    filecolor=magenta,      % color of file links
    urlcolor=cyan           % color of external links
   } 
%    colorlinks=true,       % false: boxed links; true: colored links
%    linkcolor=blue,          % color of internal links
%    citecolor=PineGreen,        % color of links to bibliography
%    filecolor=magenta,      % color of file links
%    urlcolor=cyan           % color of external links


%%%%%%%%%%%%%%%%%%%%%%%%%%%%%%%%%%%%%%%%%%%%%%%%%%%%%%%%%%%%%%%%%%%%%%%%%%%%%
% Descrition environment with reduced line spacing.
%%%%

\newenvironment{packed_description}{
\begin{description}
  \setlength{\itemsep}{1pt}
  \setlength{\parskip}{0pt}
  \setlength{\parsep}{0pt}
}{\end{description}}

%%%%%%%%%%%%%%%%%%%%%%%%%%%%%%%%%%%%%%%%%%%%%%%%%%%%%%%%%%%%%%%%%%%%%%%%%%%%%
% Section names for autorefs
%%%%

\def\chapterautorefname{Section}
\def\sectionautorefname{Section}
\def\subsectionautorefname{Section}
\def\subsubsectionautorefname{Section}

%%%%%%%%%%%%%%%%%%%%%%%%%%%%%%%%%%%%%%%%%%%%%%%%%%%%%%%%%%%%%%%%%%%%%%%%%%%%%
% Location of figures files
%%%%

\graphicspath{{./figures/}}

%%%%%%%%%%%%%%%%%%%%%%%%%%%%%%%%%%%%%%%%%%%%%%%%%%%%%%%%%%%%%%%%%%%%%%%%%%%%%
% Pi-Calculus
%%%%
\newcommand{\SET}[1]{\ensuremath{\{#1\}}} % Set braces.
\newcommand{\SUBS}[2]{\ensuremath{\SET{\raisebox{0.8pt}{\ensuremath{#1}}/\raisebox{-1.4pt}{\ensuremath{#2}}}}} % Variable substitution.

